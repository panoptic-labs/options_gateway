\documentclass[aps,preprint]{revtex4-1}

\usepackage{graphicx}
\usepackage{amsmath}
\usepackage{hyperref}

\begin{document}

\title{User Manual for the Hummingbot Gateway Panooptic Protocol Connector}

\author{Nicholas DePorzio}
\affiliation{Physical Insight LLC}

\date{\today}

\begin{abstract}
This manual provides detailed instructions on how to use . It covers installation, configuration, and usage.
\end{abstract}

\maketitle

\tableofcontents

\section{Introduction}
Welcome to the user manual for [Your Product/Software Name], developed to enable algorithmic trading in the Panoptic Protocol ~\cite{panopticXYZ}. 
This document will guide you through the features and functionalities of the product.

\section{Installation}
\subsection{System Requirements}
List the system requirements here.

\subsection{Installation Steps}
Provide step-by-step installation instructions here.

\subsubsection{Install Python Panoptic Helpers for Strategy Scripts}
\noindent 
\texttt{conda activate hummingbot}\\
\texttt{cd gateway/strategies/panopticHelpers}\\
\texttt{pip install matplotlib}\\
\texttt{pip install .}

\section{Configuration}
Explain how to configure the product after installation.

\section{Usage}
\subsection{Basic Usage}
Describe the basic usage of the product.

\section{Options Math}
\subsection{TradFi Math}
\subsection{Uniswap Math}

Critical references for this section: \cite{uniswapV3MathPrimer1,uniswapV3MathPrimer2}

Definitions of input paramters: 
\begin{itemize}
    \item $\ell$:  liquidiy [NonFungiblePositionManager$\rightarrow$positions(tokenId)]
    \item $i_u$: tickUpper [NonFungiblePositionManager$\rightarrow$positions(tokenId)]
    \item $i_l$: tickLower [NonFungiblePositionManager$\rightarrow$positions(tokenId)]
    \item $\sqrt{P}$: sqrtPriceX96 [UniswapV3Pool$\rightarrow$slot0()] 
\end{itemize}

Definitions of derived paramters: 
\begin{itemize}
    \item $i_c$: current tick 
    \item $P_c$: price at the current tick, $i_c$
    \item $P_u$: price at the tick upper location, $i_u$
    \item $P_l$: price at the tick lower location, $i_l$ 
\end{itemize}

To convert from a ``Q-number'' (e.g. `sqrtPriceX96' is a $Q=96$ number) to a decimal: 
\begin{equation}
    X_{\rm decimal} \equiv \frac{X_Q}{2^Q}
\end{equation}

To calculate the current tick from the `sqrtPriceX96' (i.e. price): 
\begin{equation}
    \frac{\sqrt{P}}{2^{96}} = 1.001^{i_c}
\end{equation}

The condition of $i_c$ being "in-range" is defined by: 
\begin{equation}
    i_l \leq i_c < i_u
\end{equation}
Note, however, in some of the example trading strategies we may reference a slightly different definition, 
where we calculate the tick locations immediately above ($i_{c+}$) and below ($i_{c-}$) the current tick 
($i_c$) and define the position as out-of-range if $i_{c-} > i_u$ or $i_{c+} < i_l$. 


To calculate token holdings FOR AN IN-RANGE POSITION: 
\begin{equation}
    x_{\rm token0} = \ell \frac{\sqrt{P_u} - \sqrt{P_c}}{\sqrt{P_u P_c}}
\end{equation}
\begin{equation}
    x_{\rm token1} = \ell (\sqrt{P_c}-\sqrt{P_l})
\end{equation}

To calculate token holdings FOR AN OUT-OF-RANGE POSITION ABOVE THE CURRENT TICK (i.e. $P_c < P_l < P_u$):
\begin{equation}
    x_{\rm token0} = \ell \frac{\sqrt{P_u} - \sqrt{P_l}}{\sqrt{P_u P_l}}
\end{equation}
\begin{equation}
    x_{\rm token1} = 0
\end{equation}

To calculate token holdings FOR AN OUT-OF-RANGE POSITION BELOW THE CURRENT TICK (i.e. $P_c > P_u > P_l$):
\begin{equation}
    x_{\rm token0} = 0
\end{equation}
\begin{equation}
    x_{\rm token1} = \ell (\sqrt{P_u}-\sqrt{P_l})
\end{equation}

\subsection{Panoptic Math}

\subsection{Advanced Features}
Detail any advanced features and how to use them.

\section{Troubleshooting}
Provide solutions to common problems users might encounter.

\section{FAQ}
List frequently asked questions and their answers.

\section{Contact Information}
Provide contact information for further support.

\appendix
\section{Appendix}
Include any additional information or resources here.

\bibliography{references}
\bibliographystyle{apsrev4-1}

\end{document}